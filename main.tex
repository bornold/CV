\documentclass[11pt,a4paper,sans]{moderncv}        % possible options include font size ('10pt', '11pt' and '12pt'), paper size ('a4paper', 'letterpaper', 'a5paper', 'legalpaper', 'executivepaper' and 'landscape') and font family ('sans' and 'roman')

% moderncv themes
\moderncvstyle{fancy}                             % style options are 'casual' (default), 'classic', 'banking', 'oldstyle' and 'fancy'
\moderncvcolor{burgundy}                               % color options 'black', 'blue' (default), 'burgundy', 'green', 'grey', 'orange', 'purple' and 'red'
\renewcommand{\familydefault}{\sfdefault}         % to set the default font; use '\sfdefault' for the default sans serif font, '\rmdefault' for the default roman one, or any tex font name
\nopagenumbers{}                                  % uncomment to suppress automatic page numbering for CVs longer than one page

% character encoding
\usepackage[utf8]{inputenc}                       % if you are not using xelatex ou lualatex, replace by the encoding you are using
%\usepackage{CJKutf8}                              % if you need to use CJK to typeset your resume in Chinese, Japanese or Korean

% adjust the page margins
\usepackage[scale=0.80]{geometry}
\setlength{\hintscolumnwidth}{5cm}                % if you want to change the width of the column with the dates
%\setlength{\makecvtitlenamewidth}{10cm}           % for the 'classic' style, if you want to force the width allocated to your name and avoid line breaks. be careful though, the length is normally calculated to avoid any overlap with your personal info; use this at your own typographical risks...

% personal data
\name{Jonas}{Bornold}
\title{CV}                               % optional, remove / comment the line if not wanted
\email{jonas.bornold@gmail.com}                               % optional, remove / comment the line if not wanted
\address{Bandgårdsgatan 21 1801}{411 12 Göteborg}{}% optional, remove / comment the line if not wanted; the "postcode city" and "country" arguments can be omitted or provided empty
\phone[mobile]{+46(0)~702~--~75~44~93}                   % optional, remove / comment the line if not wanted; the optional "type" of the phone can be "mobile" (default), "fixed" or "fax"
%\homepage{www.johndoe.com}                         % optional, remove / comment the line if not wanted
\social[linkedin]{bornold}                        % optional, remove / comment the line if not wanted
%\social[twitter]{jdoe}                             % optional, remove / comment the line if not wanted
\social[github]{bornold}                              % optional, remove / comment the line if not wanted
%\extrainfo{additional information}                 % optional, remove / comment the line if not wanted
\photo[128pt][0pt]{picture}                       % optional, remove / comment the line if not wanted; '64pt' is the height the picture must be resized to, 0.4pt is the thickness of the frame around it (put it to 0pt for no frame) and 'picture' is the name of the picture file
% \quote{Any fool can write code that a computer can understand. Good programmers write code that humans can understand. - Martin Fowler} % optional
% \quote{Code is like humor. When you have to explain it, it’s bad.}
\quote{Simplicity is prerequisite for reliability. -- E.W. Dijkstra}

% bibliography adjustements (only useful if you make citations in your resume, or print a list of publications using BibTeX)
%   to show numerical labels in the bibliography (default is to show no labels)
\makeatletter
%\renewcommand*{\bibliographyitemlabel}{\@biblabel{\arabic{enumiv}}}
\makeatother
%   to redefine the bibliography heading string ("Publications")
%\renewcommand{\refname}{Articles}

% bibliography with mutiple entries
%\usepackage{multibib}
%\newcites{book,misc}{{Books},{Others}}
%----------------------------------------------------------------------------------
%            content
%----------------------------------------------------------------------------------
\begin{document}

%\begin{CJK*}{UTF8}{gbsn}                          % to typeset your resume in Chinese using CJK
%-----       resume       ---------------------------------------------------------
\makecvtitle

\section{Arbetslivserfarenhet}
\cventry{2016--idag}{Mobile Developer}{Combitech}{Gothenburg}{}
{
  Utvecklat en mängd koncept appar till Volvo AB i både Android Native och Xamarin. 
  Detta ledde till ett två års projekt emot Volvo AB och Mack Trucks inom diagnostik och eftermarknad.
  \newline
  Resultatet blev 4 appar:
  \begin{itemize}
  \item Två Android applikationer i Volvo och Mack Trucks brand
  \item Två iOS applikation i Volvo och Mack Trucks brand
  \end{itemize}
  Dessa appar har ännu inte släppts till marknaden.
}
\section{}
\cventry{2015}{Utvecklare}{Qualisys AB}{Gothenburg}{}
{
  Projektanställd för  att utveckla en lösning att från motion capture data gå till att driva en männsklig karaktär i realtid, arbetet var en fortsättning på exjobbet.
  \newline
  Resultatet blev:
  \begin{itemize}
    \item Ett system för realtids animering av karaktärer från motion capture.
    \item Automatisk identifiering av segment av humanoida karaktärer i Unity.
    \item Demonstrationsspel i VR för att visa möjligheter med ett mo-cap system.
  \end{itemize}
  Video kan ses på 
  \url{qualisys.com/software/unity/}
}

\section{Programmeringsspråk}
\cvitemwithcomment{C\#}{Mycket erfaren}{Huvudsakliga proffesionella programmeringsspråk}
\cvitemwithcomment{Java}{Erfaren}{Proffessionellt, i utbildningen, samt i hobbyprojekt}
\cvitemwithcomment{Dart}{Grundläggande}{Används i hobbyprojekt}
\cvitemwithcomment{Python}{Grundläggande}{Används i hobbyprojekt}
\cvitemwithcomment{JavaScript}{Grundläggande}{Hobbyprojekt och utbildning}
\cvitemwithcomment{Haskell}{Grundläggande}{Användes i utbildning}
\cvitemwithcomment{C++}{Grundläggande}{Användes i utbildning}
\cvitemwithcomment{C}{Grundläggande}{Användes i utbildning}

% \clearpage
\section{Utbildning} 
\cventry{2012--2015}{M.S. in Computer Science - Algorithms, Languages and Logic}{\newline Department of Computer Science and Engineering}{Chalmers University}{}{}
% arguments 3 to 6 can be left empty
\cventry{2008--2012}{B.S. in  Software Engineering}{Chalmers University}{}{}{}  
\cventry{2008}{Data- och systemvetenskap 30hp – Fristående kurs}{Uppsala University}{}{}{}

\section{Master thesis}
\cvitem{titel}{\emph{A Real-Time Adaptation of Inverse Kinematics for Motion Capture}}
\cvitem{beskrivning}{Animating characters from motion capture in real-time} % Beskrivning på svenska
\cvitem{}{\url{publications.lib.chalmers.se/records/fulltext/219734/219734.pdf}}

\iffalse
\section{Urval av kurser}
\begin{cvcolumns}
  \cvcolumn{Algorithms}{\begin{itemize} \item Algorithms for machine learning and inference\item Data structures and algorithms\item Algorithms\end{itemize}}
  \cvcolumn{Languages}{\begin{itemize} \item Functional programming\item Parallel functional programming\item Object-oriented programming,\newline advanced course\end{itemize}}
\end{cvcolumns}
\begin{cvcolumns}
  \cvcolumn{Testing and security}{\begin{itemize}\item Language-based security\item Testing, debugging and verification\item Software engineering using formal methods\end{itemize}}
  \cvcolumn{Other}{\begin{itemize}\item Concurrent programming\item Cryptography\item Compiler construction\item Distributed applications\end{itemize}}
\end{cvcolumns}
\fi


\section{Relevanta kunskaper}
\cvitem
{Mobilt} {Xamarin, Xamarin.Forms, Android, iOS, Flutter}
\cvitem
{Certifieringar}{\href{https://devconnect.xamarin.com/profile/11738}{Xamarin Certified Mobile Professional}}
\cvitem
{DevOps} {git, bash-scripting, Google Apps Script, Azure DevOps, Cake (C\# Make)}
\cvitem
{IDEs} {Visual Studio, VS Code, Xcode, vim, Android Studio, Unity}
\cvitem
{Testing} {NUnit, XUnit, JUnit, Xamarin.UITest}


\section{Intressen}
\cvitem{Brädspel}{Perfekta blandningen av problemlösning och socialt umgänge}
\cvitem{Sci-Fi}{Stor konsument av både böcker och filmer}
\cvitem{Agrikultur}{Född och uppvuxen på gotländskt lantbruk}

\section{Repositories och länkar}
\cvitem{}{\url{github.com/bornold}}
\cvitem{}{\url{qualisys.com/software/unity}}

\section{Referenser}
\begin{cvcolumns}
  \cvcolumn{Combitech}{ \textbf{Roger Nilsson} \newline +46(0)~706~--~25~56~62 \newline \href{mailto:rogerni.nilsson@telia.com}{rogerni.nilsson@telia.com} }
  \cvcolumn{Qualisys}{ \textbf{Morgan Larsson} \newline +46(0)~707~--~76~54~71 \newline \href{mailto:morgan.larsson@qualisys.se}{morgan.larsson@qualisys.se} }
\end{cvcolumns}

\clearpage
\iffalse
    %-----       letter       ---------------------------------------------------------
    % recipient data
    \recipient{Company Recruitment team}{Company, Inc.\\123 somestreet\\some city}
    \date{January 01, 1984}
    \opening{Dear Sir or Madam,}
    \closing{Yours faithfully,}
    \enclosure[Attached]{curriculum vit\ae{}}          % use an optional argument to use a string other than "Enclosure", or redefine \enclname
    \makelettertitle

    Lorem ipsum dolor sit amet, consectetur adipiscing elit. Duis ullamcorper neque sit amet lectus facilisis sed luctus nisl iaculis. Vivamus at neque arcu, sed tempor quam. Curabitur pharetra tincidunt tincidunt. Morbi volutpat feugiat mauris, quis tempor neque vehicula volutpat. Duis tristique justo vel massa fermentum accumsan. Mauris ante elit, feugiat vestibulum tempor eget, eleifend ac ipsum. Donec scelerisque lobortis ipsum eu vestibulum. Pellentesque vel massa at felis accumsan rhoncus.

    Suspendisse commodo, massa eu congue tincidunt, elit mauris pellentesque orci, cursus tempor odio nisl euismod augue. Aliquam adipiscing nibh ut odio sodales et pulvinar tortor laoreet. Mauris a accumsan ligula. Class aptent taciti sociosqu ad litora torquent per conubia nostra, per inceptos himenaeos. Suspendisse vulputate sem vehicula ipsum varius nec tempus dui dapibus. Phasellus et est urna, ut auctor erat. Sed tincidunt odio id odio aliquam mattis. Donec sapien nulla, feugiat eget adipiscing sit amet, lacinia ut dolor. Phasellus tincidunt, leo a fringilla consectetur, felis diam aliquam urna, vitae aliquet lectus orci nec velit. Vivamus dapibus varius blandit.

    Duis sit amet magna ante, at sodales diam. Aenean consectetur porta risus et sagittis. Ut interdum, enim varius pellentesque tincidunt, magna libero sodales tortor, ut fermentum nunc metus a ante. Vivamus odio leo, tincidunt eu luctus ut, sollicitudin sit amet metus. Nunc sed orci lectus. Ut sodales magna sed velit volutpat sit amet pulvinar diam venenatis.

    Albert Einstein discovered that $e=mc^2$ in 1905.

    \[ e=\lim_{n \to \infty} \left(1+\frac{1}{n}\right)^n \]

    \makeletterclosing
\fi
%\clearpage\end{CJK*} 
\end{document}

